Logic programming is a declarative programming paradigm that has been used to advance the state of parallel programming.
Since logic programs are declarative, they are much easier to parallelize than imperative programs. First, logic programmers do not
need to use low level programming constructs such as locks or semaphores to coordinate parallel execution, because logic
systems hide such details from the programmer. Second, logic programs are easier to reason about since they are based on logical
foundations, making it easier to prove that a program is correct.

Logic based programming languages split into two main camps: forward-chaining programming languages and backwards-chaining 
programming languages. Backwards-chaining logic programs are composed of a set of rules that can be activated by inputing a query. Given a query $q(\hat{x})$, an interpreter will work backwards by matching $q(\hat{x})$ against the head of a rule. If found, the interpreter will then try to match the body of the rule, recursively, until it finds the program axioms (rules without body). If the search procedure succeeds, the interpreter finds a valid substitution for the $\hat{x}$ variables. A popular backwards-chaining programming
language is Prolog~\cite{Colmerauer:1993:BP:154766.155362}, which has been a productive research language for executing logic
programs in parallel. Researchers took advantage of Prolog's non-determinism to evaluate subgoals
in parallel with models such as \emph{or-parallelism} and \emph{and-parallelism}~\cite{Gupta:2001:PEP:504083.504085}.

In a forward-chaining logic programming language, we start with a database of facts (filled with the program's
axioms) and a set of logical rules. Then, we use the facts of the database to fire the program's rules and derive new facts that are
then added to the database. This process is repeated many times until the database reaches \emph{quiescence} and no more information can
be derived from the program.
A popular forward-chaining programming language is Datalog~\cite{Ramakrishnan93asurvey}.

In this paper, we present a new forward-chaining logic programming language called Linear Meld (LM) that is specially suited
for concurrent programming over graphs. LM differs from Datalog-like languages because it integrates both classical
logic and linear logic into the language, allowing some facts to be retracted and asserted in a logical fashion. Although most
Datalog and Prolog-like programming languages allow some kind of state manipulation~\cite{Liu98extendingdatalog}, those features
are extra-logical and do not provide a structured way to manage state like linear logic provides.

The roots of LM are the P2 system~\cite{Loo-condie-garofalakis-p2} and the original Meld~\cite{ashley-rollman-derosa-iros07wksp,ashley-rollman-iclp09}.
P2 was a Datalog-like language that mapped a computer network
to a graph, where each computer node could perform computations locally and communicate with neighbors.
Meld was itself inspired in the P2 system but adapted to the concept of massively distributed systems
made of modular robots with a dynamic topology.

LM also follows the same graph model of computation, but, instead, applies it to parallelize graph-based problems such as
graph algorithms, search algorithms and machine learning algorithms. LM programs are naturally concurrent since the graph of nodes
can be partitioned to be executed by different workers.
To realize LM, we have implemented a compiler and a virtual machine that executes LM programs on multicore machines
\footnote{Source code is available at \url{http://github.com/flavioc/meld}.}.

This paper describes the current implementation of
our virtual machine and is organized as follows: in the next section we
briefly describe LM language; in Section~\ref{virtual_machine} we present the virtual machine organization, including code organization, thread management and
execution; finally, in Section~\ref{results} we present some experimental results.