Logic programming is a declarative programming paradigm that has been used to advance the state of parallel programming.
Since logic programs are declarative, they are much easier to parallelize than imperative programs.
First, logic programs are easier to reason about since they are based on logical foundations.
Second, logic programmers do not need to use low level programming constructs such as locks or semaphores to
coordinate parallel execution, because logic systems hide such details from the programmer.

Logic programming languages split into two main fields: \emph{forward-chaining} and \emph{backwards-chaining} 
programming languages. Backwards-chaining logic programs are composed of a set of rules that can be activated by inputing a query. Given a query $q(\hat{x})$, an interpreter will work backwards by matching $q(\hat{x})$ against the head of a rule. If found, the interpreter will then try to match the body of the rule, recursively, until it finds the program axioms (rules without body). If the search procedure succeeds, the interpreter finds a valid substitution for the $\hat{x}$ variables. A popular backwards-chaining programming
language is Prolog~\cite{Colmerauer:1993:BP:154766.155362}, which has been a productive research language for executing logic
programs in parallel. Researchers took advantage of Prolog's non-determinism to evaluate subgoals
in parallel with models such as \emph{or-parallelism} and \emph{and-parallelism}~\cite{Gupta:2001:PEP:504083.504085}.

In a forward-chaining logic programming language, we start with a database of facts (filled with the program's
axioms) and a set of logical rules. Then, we use the facts of the database to fire the program's rules and derive new facts that are
then added to the database. This process is repeated until the database reaches \emph{quiescence} and no more information can
be derived from the program.
A popular forward-chaining programming language is Datalog~\cite{Ramakrishnan93asurvey}.

In this paper, we present a new forward-chaining logic programming language called Linear Meld (LM) that is specially suited
for concurrent programming over graphs. LM differs from Datalog-like languages because it integrates both classical
logic and linear logic~\cite{girard-87} into the language, allowing some facts to be retracted and asserted logically. Although most
Datalog and Prolog-like programming languages allow some kind of state manipulation~\cite{Liu98extendingdatalog}, those features
are extra-logical, reducing the advantages brought forward by logic programming.
In LM, since mutable state remains within the logical framework, we can reason logically about LM programs.

The roots of LM are the P2 system~\cite{Loo-condie-garofalakis-p2} and the original Meld~\cite{ashley-rollman-derosa-iros07wksp,ashley-rollman-iclp09}.
P2 is a Datalog-like language that maps a computer network
to a graph, where each computer node performs local computations and communicates with neighbors.
Meld is itself inspired by the P2 system but adapted to the concept of massively distributed systems
made of modular robots with a dynamic topology.
LM also follows the same graph model of computation, but, instead, applies it to parallelize graph-based problems such as
graph algorithms, search algorithms and machine learning algorithms. LM programs are naturally concurrent since the graph of nodes
can be partitioned to be executed by different workers.

To realize LM, we have implemented a compiler and a virtual machine that executes LM programs on multicore machines\footnote{Source code is available at \url{http://github.com/flavioc/meld}.}.
We have implemented several parallel algorithms, including: belief propagation~\cite{Gonzalez+al:aistats09paraml},
belief propagation with residual splash~\cite{Gonzalez+al:aistats09paraml}, PageRank, graph coloring,
N-Queens, shortest path, diameter estimation, map reduce, quick-sort, neural network training, among others.

\iffalse
There are also non logical based systems intended to solve graph-based problems such as Dryad, Pregel or GraphLab.
The Dryad system~\cite{Isard:2007:DDD:1272996.1273005} is a framework that combines computational vertices
with communication channels (edges) to form a data-flow graph. 
The Pregel system~\cite{Malewicz:2010:PSL:1807167.1807184} is also graph-based, although programs must be represented
as a sequence of iterations where each iteration is composed of computation and message passing.
Finally, GraphLab~\cite{GraphLab2010} is a library for developing parallel (graph-based) machine learning algorithms in C++
and provides several schedulers to dictate the order of node execution.
%While Pregel uses message passing, GraphLab allows nodes to have read/write access to different scopes through different concurrent access models in order to balance performance and data consistency. Each consistency model provides different guarantees that are suited to multiple classes of algorithms.
\fi

As a forward-chaining linear logic programming language, LM shares similarities with Constraint Handling Rules (CHR)~\cite{Betz:2005kx,DBLP:journals/corr/abs-1006-3039}.
CHR is a concurrent committed-choice constraint language used to write constraint solvers. A CHR program is a set of rules and
a set of constraints (which can be seen as facts). Constraints can be consumed or generated during the application of rules.
Some optimization ideas used in LM such as join optimizations and using different data structures for indexing facts
are inspired by research done in CHR~\cite{DBLP:journals/corr/cs-PL-0408025}.

This paper describes the current implementation of
our virtual machine and is organized as follows. First, we
briefly introduce the LM language. Then, we present an overview of the virtual machine, including code organization,
thread management, rule execution and database organization. Finally, we present preliminary results and
outline some conclusions.
